\documentclass[main.tex]{subfiles}

\begin{document}

\chapter{为政第二}

\paragraph{2.1}
子曰:“为政以德,譬如北辰,居其所而众星共之。”

\paragraph{2.2}
子曰:“《诗》三百,一言以蔽之,曰:’思无邪’。”

\paragraph{2.3}
子曰:“道之以政,齐之以刑,民免而无耻。道之以德,齐之以礼,有耻且格。”

\paragraph{2.4}
子曰:“吾十有五而志于学,三十而立,四十而不惑,五十而知天命,六十而耳顺,七十而从心所欲不逾矩。”

\paragraph{2.5}
孟懿子问孝,子曰:“无违。”樊迟御,子告之曰:“孟孙问孝于我,我对曰:’无违。’”
樊迟曰:“何谓也?”子曰:“生,事之以礼。死,葬之以礼,祭之以礼。”

\paragraph{2.6}
孟武伯问孝,子曰:“父母唯其疾之忧。”

\paragraph{2.7}
子游问孝,子曰:“今之孝者,是谓能养。至于犬马,既能有养。不敬,何以别乎?”

\paragraph{2.8}
子夏问孝,子曰:“色难。有事,弟子服其劳。有酒食,先生馔。曾是以为孝乎?”

\paragraph{2.9}
子曰:“吾与回言,终日不违,如愚。退而省其私,亦足以发。回也不愚。”

\paragraph{2.10}
子曰:“视其所以,观其所由,擦其所安,人焉廋哉?人焉廋哉?”

\paragraph{2.11}
子曰:“温故而知新,可以为师矣。”

\paragraph{2.12}
子曰:“君子不器。”

\paragraph{2.13}
子贡问君子,子曰:“先行其言而后从之。”

\paragraph{2.14}
子曰:“君子周而不比,小人比而不周。”

\paragraph{2.15}
子曰:“学而不思,则罔。思而不学,则殆。”

\paragraph{2.16}
子曰:“攻乎异端,斯害也已。”

\paragraph{2.17}
子曰:“由,诲女知之乎!知之为知之,不知为不知,是知也。”

\paragraph{2.18}
子张学干禄。子曰:“多闻阙疑,慎言其余,则寡尤。多见阙殆,慎行其余,则寡悔。言寡尤,行寡悔,禄在其中矣。”

\paragraph{2.19}
哀公问曰:“何为则民服?”孔子对曰:“举直错诸枉,则民服。举枉错诸直,则民不服。”

\paragraph{2.20}
季康子问:“市民敬忠以劝,如之何?”子曰:“临之以庄,则敬。孝慈,则忠。举善而教不能,则劝。”

\paragraph{2.21}
或谓孔子曰:“子奚不为政?”子曰:“《书》云:’孝乎惟孝,友于兄弟。’施于有政,是亦为政,奚其为为政?”

\paragraph{2.22}
子曰:“人而无信,不知其可也。大车无輗,小车无(车兀),其何以行之哉?”

\paragraph{2.23}
子张问:“十世可知也?”子曰:“殷因于夏礼,所损益可知也。周因于殷礼,所损益可知也。其或继周者,虽百世可知也。”

\paragraph{2.24}
子曰:“非其鬼而祭之,谄也。见义不为,无勇也。”

\end{document}