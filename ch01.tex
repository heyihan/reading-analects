\documentclass[main.tex]{subfiles}

\begin{document}

\chapter{中国文化}

\section{儒家}

\subsection{论语}

\subsubsection{第一篇学而}
\paragraph{1.1}
子曰:“学而时习之,不亦说乎?有朋自远方来,不亦乐乎?人不知而不愠,不亦君子乎?”

\paragraph{1.2}
有子曰:“其为人也孝弟,而好犯上者,鲜矣。不好犯上,而好作乱者,未之有也。君子务本,本立而道生。孝弟也者,其为仁之本与?”

\paragraph{1.3}
子曰:“巧言令色,鲜矣仁。”

\paragraph{1.4}
曾子曰:“吾日三省吾身。为人谋,而不忠乎?与朋友交,而不信乎?传,不习乎?”

\paragraph{1.5}
子曰:“道千乘之国,敬事而信,节用而爱人,使民以时。”

\paragraph{1.6}
子曰:“弟子入则孝,出则弟,谨而信,泛爱众,而亲仁。行有余力,则以学文。”

\paragraph{1.7}
子夏曰:“贤贤易色,事父母能竭其力,事君能致其身,与朋友交,言而有信,虽曰未学,吾必谓之学矣。”

\paragraph{1.8}
子曰:“君子不重则不威。学则不固。主忠信。无友不如己者。过则勿惮改。”

\paragraph{1.9}
曾子曰:“慎终追远,民德归厚矣。”

\paragraph{1.10}
子禽问于子贡曰:“夫子至于是邦也,必闻其政。求之与?抑与之与?”
子贡曰:“夫子温、良、恭、俭、让以得之。夫子之求之也,其诸异乎人之求之与!”

\paragraph{1.11}
子曰:“父在观其志,父没观其行。三年无改于父之道,可谓孝矣。”

\paragraph{1.12}
有子曰:“礼之用,和为贵。先王之道,斯为美。小大由之。有所不行。和和而和,不以礼节之,亦不可行也。”

\paragraph{1.13}
有子曰:“信近于义,言可复也。恭近于礼,远耻辱也。因不失其亲,亦可宗也。”

\paragraph{1.14}
子曰:“君子食无求饱,居无求安,敏于事而慎于言,就有道而正焉,可谓好学也已。”

\paragraph{1.15}
子贡曰:“贫而无谄,富而无骄,何如?”子曰:“可也。未若贫而乐,富而好礼者也。”
子贡曰:“《诗》云:‘如切如磋,如琢如磨。’其斯之谓与?”子曰:“赐也!始可与言诗已矣。告诸往而知来者。”

\paragraph{1.16}
子曰:“不患人之不己知,患不知人也。”

\subsubsection{第二篇为政}
\paragraph{2.1}
子曰:“为政以德,譬如北辰,居其所而众星共之。”

\paragraph{2.2}
子曰:“《诗》三百,一言以蔽之,曰:’思无邪’。”

\paragraph{2.3}
子曰:“道之以政,齐之以刑,民免而无耻。道之以德,齐之以礼,有耻且格。”

\paragraph{2.4}
子曰:“吾十有五而志于学,三十而立,四十而不惑,五十而知天命,六十而耳顺,七十而从心所欲不逾矩。”

\paragraph{2.5}
孟懿子问孝,子曰:“无违。”樊迟御,子告之曰:“孟孙问孝于我,我对曰:’无违。’”
樊迟曰:“何谓也?”子曰:“生,事之以礼。死,葬之以礼,祭之以礼。”

\paragraph{2.6}
孟武伯问孝,子曰:“父母唯其疾之忧。”

\paragraph{2.7}
子游问孝,子曰:“今之孝者,是谓能养。至于犬马,既能有养。不敬,何以别乎?”

\paragraph{2.8}
子夏问孝,子曰:“色难。有事,弟子服其劳。有酒食,先生馔。曾是以为孝乎?”

\paragraph{2.9}
子曰:“吾与回言,终日不违,如愚。退而省其私,亦足以发。回也不愚。”

\paragraph{2.10}
子曰:“视其所以,观其所由,擦其所安,人焉廋哉?人焉廋哉?”

\paragraph{2.11}
子曰:“温故而知新,可以为师矣。”

\paragraph{2.12}
子曰:“君子不器。”

\paragraph{2.13}
子贡问君子,子曰:“先行其言而后从之。”

\paragraph{2.14}
子曰:“君子周而不比,小人比而不周。”

\paragraph{2.15}
子曰:“学而不思,则罔。思而不学,则殆。”

\paragraph{2.16}
子曰:“攻乎异端,斯害也已。”

\paragraph{2.17}
子曰:“由,诲女知之乎!知之为知之,不知为不知,是知也。”

\paragraph{2.18}
子张学干禄。子曰:“多闻阙疑,慎言其余,则寡尤。多见阙殆,慎行其余,则寡悔。言寡尤,行寡悔,禄在其中矣。”

\paragraph{2.19}
哀公问曰:“何为则民服?”孔子对曰:“举直错诸枉,则民服。举枉错诸直,则民不服。”

\paragraph{2.20}
季康子问:“市民敬忠以劝,如之何?”子曰:“临之以庄,则敬。孝慈,则忠。举善而教不能,则劝。”

\paragraph{2.21}
或谓孔子曰:“子奚不为政?”子曰:“《书》云:’孝乎惟孝,友于兄弟。’施于有政,是亦为政,奚其为为政?”

\paragraph{2.22}
子曰:“人而无信,不知其可也。大车无輗,小车无(车兀),其何以行之哉?”

\paragraph{2.23}
子张问:“十世可知也?”子曰:“殷因于夏礼,所损益可知也。周因于殷礼,所损益可知也。其或继周者,虽百世可知也。”

\paragraph{2.24}
子曰:“非其鬼而祭之,谄也。见义不为,无勇也。”

\subsubsection{人物}

孔子弟子有多少,著名的弟子有哪些,传承如何?
\begin{description}
    \item[颜回] 字渊,孔子早年学生,最为孔子所深爱
    \item[有子] 名若,孔子晚年学生,孔门弟子学问最高的几个人
    \item[曾子] 名参,孔子晚年学生,孔门弟子学问最高的几个人,孟子称曾子“守约”
    \item[子路] 仲由,字子路,孔子早年学生
    \item[子夏] 卜商,字子夏,孔子晚年学生
    \item[子游] 言偃,字子游,孔子晚年学生
    \item[子张] 瑞孙师,字子张,孔子晚年学生
    \item[子禽] 陈亢即原亢,字子禽
    \item[子贡] 端木赐,字子贡
    \item[樊迟] 名须
    \item[孟懿子] 氏仲孙,名何忌,鲁国大夫,三家之一。孔子早期学生,但因孔子主堕三家之都,而何忌首抗命,后人不列为孔门弟子
\end{description}


\section{道家}

\end{document}