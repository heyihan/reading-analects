\documentclass[main.tex]{subfiles}

\begin{document}

\chapter{学而第一}

\paragraph{1.1}
子曰:“学而时习之,不亦说乎?有朋自远方来,不亦乐乎?人不知而不愠,不亦君子乎?”

\paragraph{1.2}
有子曰:“其为人也孝弟,而好犯上者,鲜矣。不好犯上,而好作乱者,未之有也。君子务本,本立而道生。孝弟也者,其为仁之本与?”

\paragraph{1.3}
子曰:“巧言令色,鲜矣仁。”

\paragraph{1.4}
曾子曰:“吾日三省吾身。为人谋,而不忠乎?与朋友交,而不信乎?传,不习乎?”

\paragraph{1.5}
子曰:“道千乘之国,敬事而信,节用而爱人,使民以时。”

\paragraph{1.6}
子曰:“弟子入则孝,出则弟,谨而信,泛爱众,而亲仁。行有余力,则以学文。”

\paragraph{1.7}
子夏曰:“贤贤易色,事父母能竭其力,事君能致其身,与朋友交,言而有信,虽曰未学,吾必谓之学矣。”

\paragraph{1.8}
子曰:“君子不重则不威。学则不固。主忠信。无友不如己者。过则勿惮改。”

\paragraph{1.9}
曾子曰:“慎终追远,民德归厚矣。”

\paragraph{1.10}
子禽问于子贡曰:“夫子至于是邦也,必闻其政。求之与?抑与之与?”
子贡曰:“夫子温、良、恭、俭、让以得之。夫子之求之也,其诸异乎人之求之与!”

\paragraph{1.11}
子曰:“父在观其志,父没观其行。三年无改于父之道,可谓孝矣。”

\paragraph{1.12}
有子曰:“礼之用,和为贵。先王之道,斯为美。小大由之。有所不行。和和而和,不以礼节之,亦不可行也。”

\paragraph{1.13}
有子曰:“信近于义,言可复也。恭近于礼,远耻辱也。因不失其亲,亦可宗也。”

\paragraph{1.14}
子曰:“君子食无求饱,居无求安,敏于事而慎于言,就有道而正焉,可谓好学也已。”

\paragraph{1.15}
子贡曰:“贫而无谄,富而无骄,何如?”子曰:“可也。未若贫而乐,富而好礼者也。”
子贡曰:“《诗》云:‘如切如磋,如琢如磨。’其斯之谓与?”子曰:“赐也!始可与言诗已矣。告诸往而知来者。”

\paragraph{1.16}
子曰:“不患人之不己知,患不知人也。”

\end{document}